\documentclass{scrartcl}
\usepackage{riley}
\usepackage{import}

\newcommand{\ball}{B}
\begin{document}
\begin{defn}[nowhere dense]\label{nowhere-dense}
  A nowhere dense set's closure has empty interior.
\end{defn}
\begin{defn}[meager]\label{meager}
  A meager set is countable union of \nameref{nowhere-dense}~sets
\end{defn}
\begin{theorem}[Baire category]\label{baire category}
  If \(X\) is a \emph{completely metrizable} space,
  \begin{enumerate}
  \item countable intersections of \emph{open} dense subsets are dense,
  \item \(X\) is not \nameref{meager}.
  \end{enumerate}
\end{theorem}
\begin{proof} \
  \begin{enumerate}
  \item
    Consider \(U_i\), a sequence of open dense sets. It suffices to show a nonempty open set~\(W\)
    meets \(\bigcap U_i\). The proof proceeds by inductively\footnote{In a general (nonseparable) metric space, this requires the axiom of dependent choice.} nesting open balls inside successive \(W\cap U_i\). By assumption of density, \(W\cap U_1\) is nonempty. It is also open, so there is some \(\ball_1= \ball\paren{x_1,r_1}\) with \(\closure{\ball_1}\subseteq W\cap U_0\). Observe \(\ball_1\cap U_1\) is also open; therefore, find a \(\ball_2=\ball(x_2,r_2)\) where \(\closure{\ball_2}\subseteq \ball_1\cap U_1\) and proceed inductively: \(\ball_{n+1}=\ball(x_{n},r_{n})\) where \(\closure{\ball_{n+1}}\subseteq \ball_n\cap U_n\). These are nested and
    \[
      x_n \in \grayunderbrace{\ball_n}{\subseteq U_n}\subseteq \grayunderbrace{\ball_{n-1}}{\subseteq U_{n-1}} \subseteq  \grayunderbrace{\ball_2}{\subseteq U_2} \subseteq \grayunderbrace{\ball_1}{\subseteq U_1} \subseteq W
    \]
    \begin{center}
      \subimport{figs/}{baire.pdf_tex}
    \end{center}
    If \(r_n\) are selected such that \(r_n\to 0\), \eg, by requiring \(r_n < 2^{-n}\) at each step of the induction,
    then \(x_n\) is Cauchy. By completeness, there is a limit; call it \(x\). But because \(x_i\) is eventuall in \(\ball_n\), \(x\in \closure{\ball\paren{x_n,r_n}}\subseteq \ball\paren{x_{n-1},r_{n-1}}\) for all \(n\). Then
    \[
      x \in \dots \grayunderbrace{\ball_n}{\subseteq U_n}\subseteq \grayunderbrace{\ball_{n-1}}{\subseteq U_{n-1}} \subseteq  \grayunderbrace{\ball_2}{\subseteq U_2} \subseteq \grayunderbrace{\ball_1}{\subseteq U_1} \subseteq W.
    \]
    Therefore, \(x\in \bigcap U_n\cap W\).
  \item By the previous result, it suffices to show the complement of any \nameref{nowhere-dense} set contains an open, dense set.
    Suppose \(N\) is nowhere dense. Then \(N^C\) is dense; hence, \(\interior{\paren[]{N^C}}\) is open and dense.
  \end{enumerate}
\end{proof}
\begin{theorem}[Open mapping theorem]
  Surjective Banach-morphisms are open.
\end{theorem}
\begin{proof}
  Call the morphism \(T:X\to Y\). Let \(\ball_r=\ball\paren{0,r}\in X\). Because linear maps commute with translations and dilations, it suffices to show that \(0\) is an interior point of \(T(B_1)\). By surjectivity,
  \(Y = \bigcup_{n=1}^\infty TB_n\). The map \(n\blank: \paren{Y,TB_1}\to \paren{Y,TB_n}\) is a homeomorphism for \(n>0\). Therefore, \(B_1\) cannot be \nameref{nowhere-dense}: if it were, \(\ball_n\) would be too, making \(Y\) \nameref{meager}, which contradicts \namecref{baire category}.

  The closure~\(\closure{T\ball_1}\) contains an interior point \(y_0\). For some \(r\), \(\ball(y_0,4r)\subseteq \closure{T\ball_1}\). Then there is some \(y_1=Tx_1\in T\ball_1\) within \(2r\) of \(y_0\). In particular, its neighborhood \(\ball(y_1,2r)\subseteq \ball(y_0,4r)\subseteq \closure{T\ball_1}\)
  \begin{center}
    \import{figs/}{openmapping.pdf_tex}
  \end{center}
  If \(\norm y < 2r\), then \(y_1-y\in \closure{T\ball_1}\) so \[y=y_1-\paren{y_1-y} = Tx_1 - (y_1-y)\in \closure{T(x+\ball_1)}\subseteq \closure{T\ball_2}.\] Halving this shows \(\norm y < r \implies y\in \closure{T\ball_1}\); repeatedly halving shows
  \begin{equation}
    \norm y < r2^{-n} \implies y\in \closure{T\ball_{2^{-n}}}.\label{eq:openmap-small enough to be in closed image}
  \end{equation}

Now, chase an arbitrary \(y\) in \(\set{y:\norm y < \sfrac r 2}\) with a sequence of \(Tx_i\). \Cref{eq:openmap-small enough to be in closed image} guarantees an \(x_1\in \ball_1\) such that \(\norm{y-Tx_1} < \sfrac r 4\), and, by induction, a sequence such that \(\norm{y-\sum_1^n Tx_i} < r2^{n-1}\) with \(x_i\in\ball_i\). The series \(\sum x_i\) is hence absolutely convergent; by completeness, it has some limit \(x\). But
\[
  \norm{x} \leq \sum_i \norm{x_i} = \sum_i 2^{-i} = 1
\]
with \(Tx=\lim T\paren{\sum x_i}=y\) by construction. Therefore \(y\in T\ball_1\) itself, not just its closure. Hence
\[
  \grayunderbrace{\set{y: \norm y < \sfrac r 2}}{\text{open nhood of } 0} \subseteq T\ball_1
\]
\end{proof}
Because the inverse of an open map is continuous,
\begin{cor}
  A bijective Banach-morphism is an isomorphism.
\end{cor}
\end{document}